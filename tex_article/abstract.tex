\begin{abstract}

Within this project we implement a program that uses the SIR model to simulate the outbreak of a pandemy. The program can also operate backwards and estimate the SIR parameters of a given set of data by using the least squares method.

As variations on the theme we implement a forecasting program based on the logistic model, as well as a real-time graphic simulation of the spreading of a flu among a closed group of people.

All  the code is written in C++.

External libraries needed to properly run the program (some of them must be installed onto the user's computer):
\begin{itemize}
\item \href{https://www.bfgroup.xyz/Lyra/}{Lyra};
\item \href{https://www.sfml-dev.org/}{SFML};
\item \href{https://en.wikipedia.org/wiki/Doctest}{Doctest};
\item \href{https://cmake.org/}{CMake (optional)};
\end{itemize}

The work is divided among three directories:
\begin{description}
\item[mandatory\_SIR\_model] It contains the headers and source code of the main program.
\item[logistic\_model] It contains the headers and the source code of the logistic model program.
\item[graphic\_simulation] It contains the headers and the source code of the graphic simulation program.
\end{description}
Plus a directory, \verb!files!, with the .clang-format and other files. 

More about the SIR model at \cite{wikipedia : SIR}, \cite{Gaeta : Modelli matematici}, \cite{Exactly solvable SIR}.
\end{abstract}
